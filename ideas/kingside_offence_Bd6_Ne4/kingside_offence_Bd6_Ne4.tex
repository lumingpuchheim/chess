\minisec{Petrosian--Spassky, 1969}
\newchessgame[
id=A,
moveid=14b,
setwhite={pa2, pb4, pd4, pe3, pf2, pg2, ph2, kg1, qd2, rb1, rc1, bd3, nc3, nf3},
addblack={pa7, pb6, pc6, pd5, pf7, pg7, ph6, kg8, qd8, ra8, re8, bb7, bf6, nd7}]
\chessboard

In the game Spassky played a5, intending a queen side offense. 

``Spassky initiates counterplay on the queenside, but a different plan deserved serious attention: the inactive bishop on f6 could be transferred via e7 to d6 where it 
could be used for operations on the kingside.'' [Bondarevsky]

``Black had a plan available that is typical of such positions: transfer the bishop to d6 and the kight via f6 to e4.
What could White do to oppose this? Pushing the a-pawn to a5 gives very little, while b4-b5 always comes up against ...c5. White would have to carry out a central break with e3-e4, but 
this would lead to simplification and offer few chances of success.'' [Boleslavsky]
