
\newcommand\getmovestyle[1]{%
\ifthenelse
  {\equal{#1}{N}}%knight move
  {\def\mymovestyle{[clockwise=false,style=knight]curvemove}}%
  {\ifthenelse
    {\equal{#1}{}}% castling
    {\def\mymovestyle{curvemove}}%
    {\def\mymovestyle{straightmove}}}}%

In chess, a minority attack is the advancement of one's pawns on the side of the board where one has fewer pawns than their opponent, intending to use their minority to strategically provoke a weakness (i.e, an isolated or backward pawn) in the opponent's pawn structure. The minority attack is a common middlegame plan that can be played in many pawn structures. The name might be misleading, as the "attack" does not involve tactics planned to produce checkmate or significant material gain, but rather a strategical and structural advantage for the attacking player.

The minority attack can be strengthened by the moving of one or both rooks to the files where the attacking player intends to advance their pawns, planning prophylactically for the opening of the files. Common openings that result in pawn structures where a minority attack is effective include the Queen's Gambit Declined and the Caro-Kann Defense. The minority attack occurs most commonly on the queenside, as players commonly castle kingside in openings where a minority attack is effective, and the advancement of the pawns on the side of the castled king is widely considered to severely weaken the king's safety.

\minisec{Basic Form}
White thrusts a- and b-pawns to create a weakness for black on c6.

\newchessgame[
id=A,
moveid=1w,
setwhite={pa2, pb2, pd4, pe3, pf2, pg2, ph2},
addblack={pa7, pb7, pc6, pd5, pf7, pg7, ph7}]
\mainline{1. b4 ... 2. a4 ... 3. b5}

\setchessboard{shortenend=5pt,color=blue}%
\chessboard[lastmoveid=A,setfen=\xskakget{nextfen},
moveid=1w,
pgfstyle=straightmove,
markmove=\xskakget{movefrom}-\xskakget{moveto},
stepmoveid=2,color=blue,
markmove=\xskakget{movefrom}-\xskakget{moveto},
stepmoveid=2,color=blue,
markmove=\xskakget{movefrom}-\xskakget{moveto},
pgfstyle=color,
opacity=0.5,
color=red,
markfield={c6}]

\minisec{Capablanca--H.Golombek, 1939}
%\centering
\newchessgame[
id=main,
moveid=23w,print,
showmover,
mover=b,% has no effect
castling=Q,enpassant=a3,
setwhite={pa2,pb4,pd4,pe3,pf2,pg2,ph3,qd3,kg1,rb1,rc1,na4},
addblack={pa7,pb7,pc6,pd5,pf7,pg6,ph6,kg8,qd6,ra8,re8,ng7}]

\mainline{23.b5 cxb5}

White continues the minority attack and favorably changes the pawn structure. If black allows white to capture on c6, then he will have a backward c6-pawn. If black captures on b5, white will recapture with the queen and can target the isolated d5 and b7 pawn.

\mainline{24.Qxb5 Ne6}

\chessboard

\mainline{25. Nc3!}

Much better than 
\variation{25. Qxb7} which gives black chances after \variation{25... Reb8 26. Qc6 Qxc6 27. Rxb8+ Rxb8 28. Rxc6 Rb1+}.

\mainline{25... Red8 26. Qxb7 Qa3 27.Nxd5 Qxa2}

\chessboard

\mainline{28.Nb4 Qa4 29. Nc6}
1-0

For white not only threatens the rook but also the queen with \variation{30. Ra1}.

\minisec{Black's Minority Attack in Sicilian Defence}
The Sicilian Defence minority attack was known in the 1920s, but not so much so that books made mention of it. Later, with the Sicilian Defence explosion, that minority attack became one of the main reasons why White felt obliged to launch attacks at an early stage against Black's position \cite{book:secrets_of_modern_chess_strategy}.

\newgame
\mainline{1.e4 c5 2.Nf3 d6 3.d4 cxd4 4.Nxd4 Nf6 5.Nc3 e6 6.Be2 Nc6 7.Be3 Be7 8.O-O O-O 9.f4 Bd7 10.Nb3 a6 11.Bf3 Rb8 12.Qe1 b5 13.Rd1 b4 14.Ne2 e5 15.f5 Na5 }

\chessboard

Black wants to tie White down with moves like \variation{...Bb5} and \variation{...Nc4}, followed by \variation{...a4}, a typical minority attack as White's plan in Queen's Gambit Exchange Variation.

\mainline{16.Nxa5 Qxa5 17.g4 Rfc8 18.g5 Ne8 19.Rd2 Qxa2 20.Ng3 Bf8 21.Nh5 Qxb2 22.Qg3 Rc3 23.Bg4 Qa3 24.Re1 b3 25.g6 fxg6 26.fxg6 Rxe3 27.gxh7+ Kxh7 28.Rxe3 Bxg4 29.Qxg4 Qc1+ 30.Qd1 b2 31.Re1 Qxd1 32.Rdxd1 a5 33.Ng3 a4 34.Ne2 Rc8 35.c3 a3 36.Rb1 Rb8 37.Nc1 bxc1=Q 38.Rexc1 Ra8 39.Ra1 Nf6 40.Ra2 Nxe4 41.Rca1 d5 42.Rc1 Rc8 43.Rac2 Rxc3 44.Rxc3 Nxc3 45.Rxc3 a2 46.Rc1 Bc5+ 47.Kg2 Bd4 48.Kf3 a1=Q 49.Rxa1 Bxa1 50.Kg4 Kg6  0-1}




