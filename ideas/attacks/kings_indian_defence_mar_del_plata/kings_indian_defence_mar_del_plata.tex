\newgame
\newchessgame[
id=A,
moveid=1w,
]

\mainline{1. d4 Nf6 2. c4 g6 3. Nc3 Bg7 4. e4 d6 5. Be2 O-O 6. Nf3 e5 7. O-O Nc6 8. d5 Ne7}
\savegame{mar_del_plata}
\chessboard

``The variation of the Kings's Indian is often called the Mar del Plata Variation and it has been a key line for half a 
century. Here the traditional theme of Whites's queenside play versus Black's kingside attack is played out with
particular clarity. Despite decades of analysis, the final verdict remains unclear, but today's leading
players do not like the committal nature of Black's strategy and these days it is notoften seen at the highest level\cite{book:grandmaster_chess_move_by_move}''

\minisec{White Plans: possible ideas}
\begin{itemize}
    \item{Pawn storm on the queenside.}
    \item{Aim to penetrate on the queenside}
    \item{Exchange the light square bishop}

    Black needs this piece in a great many lines to sacrifice on h3
\end{itemize}   

\minisec{Black Plans: pawn storm aiming checkmate}
\begin{itemize}
    \item{Pawn storm on the kingside with f5, f4 and g5 \footnote{``In the Second World War, the Germans and then also the Russians employed the following method of warfare: after getting drunk before a battle, silently with their weapons at the ready, standing up straight and making no effort to conceal themselves, they would automatically advance towards the entrenched enemy \cite{book:my_best_games_korchnoi}''} }
    \item{Manoeuvre with \symrook{} f7 (aiming g7) and \symbishop{} f8 (protecting d6, make space for the rook)}
    \item{Avoid exchanging the light square bishop}

    Black needs this piece in a great many lines to sacrifice on h3
\end{itemize}  




