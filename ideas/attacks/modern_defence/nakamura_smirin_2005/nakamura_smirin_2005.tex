\minisec{Nakamura-Smirin,2005}

\newgame
\newchessgame[
id=A,
moveid=1w,
]

\mainline{1. e4 g6 2. d4 Bg7 3. Nc3 d6 4. f4 Nf6 5. Nf3 O-O 6. e5 Nfd7 7. h4 c5 8. h5 cxd4 9. hxg6 dxc3 }

\chessboard
\savegame{nakamura_smirin_2005_move10}

\mainline{10. gxf7 Rxf7 11. Bc4 Nf8 12. Ng5} 

White can already force a draw here with 

\variation{12. Bxf7 Kxf7 13. Ng5 Kg8 14. Qh5 h6 15. Qf7 Kh8 16.Qb3 Qa5 17. Nf7 Kh7 18. Ng5 Kh8 19. Nf7 Kh7} 

White gives perpetual check. Hikaru wants more!

\mainline{12...e6 13. Nxf7 cxb2}?!

The main defence is \variation[outvar]{13...Kxf7 14. Qh5+ Kg8 15. Bd3 h6 16. Rh4 dxe5 17. Rg4 e4 18. f5 exf5} 

(\variation[invar]{18...exd3 19. Rxg7+ Kxg7 20. Bxh6+ Kg8 21. Qg4+ Kf7 22. Qg7+ Ke8 23. Qxf8+ Kd7 24. fxe6+ }) 

\variation[outvar]{19. Rxg7+ Kxg7 20. Bxh6+ Kg8 21. O-O-O Be6 22. Bc4 cxb2 23. Kb1 Qf6 24. Bxe6 Nxe6 25. Qe8+ Nf8 26. Bxf8 Qf7 27. Qc8 Nd7 28. Qxa8 Nxf8 29. Qxa7} White should win.


Now we go back to the main game.

\chessboard

\mainline{14. Bxb2 Qa5+ 15. Kf1 Kxf7 16. Qh5+ Kg8 17. Bd3 Qb4 18. Rb1 Bd7 19.c4}
White protects \symrook{} b1, frees the \symbishop{}b2 and at the same time cuts the black queen off from the king side \cite{book:fighting_chess_with_hikaru}.

\mainline{19...Qd2 20. Bxh7+ Nxh7 21. Qxh7+ Kf8 22. Rh4}

There is nothing more that can be done against \variation{23.Rg4} therefore Black resigned.



