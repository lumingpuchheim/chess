\minisec{Botvinnik--Keres, 1953}
%\centering
\newchessgame[
id=main
]

\mainline{1.d4 Nf6 2.c4 e6 3.Nc3 d5 4.cxd5 exd5 5.Bg5 Be7 6.e3 O-O 7.Bd3
Nbd7 8.Qc2 Re8 9.Nge2 Nf8 10.O-O c6 11.Rab1 Bd6 }

\chessboard

\mainline{12.Kh1 Ng6}

Prophylactic move: moves the king away from potential \symbishop  xh2 check. Minority attack loses a piece
\variation{ 12. b4 Bxh2+ 13. Kxh2 Ng4+ 14. Kg1 Qxg5 15. Nf4 Qh6 16. Nh3 Qd6 17. Nf4 g5 }.

\chessboard

\mainline{13.f3}

\chessboard

White has two options, minority attack or central play. 
Minority attack still fails tactically: \varation{13. b4 h6 14. Bxf6 Qxf6 15. b5 Nh4 16. Ng1 Bxh2 17. bxc6 bxc6 18. Kxh2 Qg5 19. g3 Qh5 20. gxh4 Qxh4+ 21. Kg2 Qg4+ 22. Kh1} =, since Black has perpetual check.

White should play for center also because White's pieces are better placed and coordinated to support a central breakthrough.

f3 is a good move because it restricts Black's options: the square g4 is now forbidden, and e4 cannot be occupied by a black piece.
\mainline{13...Be7}

\varation{13...h6 14. Bxf6 Qxf6 15. e4 Qh4 16. e5} White closes the b8-h2 digonal to stop Black's attack at once.

\mainline{14.Rbe1 Nd7 15.Bxe7 Rxe7 16.Ng3 Nf6 }

Note how White brings all  the pieces to attack. The b1 rook and the e2 knight all joins the party.

\mainline{17.Qf2} 

\lastmove{} is necessary because otherwise d4 pawn hangs.

\mainline{17...Be6
18.Nf5 Bxf5 19.Bxf5 Qb6} 

\chessboard

Everything is ready. White cannot further improve his position. It is time to strike with e4!

\mainline{20.e4 dxe4 21.fxe4 Rd8 }

White has completed his plan to push e3-e4 breakthrough and is certainly better. The rest is not part of the breakthrough strategy, but still very instructive.

\chessboard

\mainline{22.e5 Nd5
23.Ne4 Nf8 24.Nd6 Qc7 25.Be4 Ne6 26.Qh4 g6 27.Bxd5 cxd5 28.Rc1
Qd7 29.Rc3 Rf8 30.Nf5 Rfe8 31.Nh6+ Kf8 32.Qf6 Ng7 33.Rcf3 Rc8
34.Nxf7 Re6 35.Qg5 Nf5 36.Nh6 Qg7 37.g4}

1-0

\chessboard
