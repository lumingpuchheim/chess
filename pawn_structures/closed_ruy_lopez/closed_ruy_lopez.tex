The Closed Ruy Lopez is a classic chess opening that arises after the moves:

\newgame
\newchessgame[
id=A,
moveid=1w,
]

\mainline{1. e4 e5 2. Nf3 Nc6 3. Bb5 a6 4. Ba4 Nf6 5. O-O Be7 6. Re1 b5 7. Bb3 d6 8. c3 O-O 9. h3}

\chessboard

This variation is one of the most deeply analyzed and strategically rich openings in chess history. Named after the 16th-century Spanish priest Ruy López de Segura, it has stood the test of time and remains a popular choice at all levels of play, from club games to World Championship matches.

The Closed Ruy Lopez typically leads to slow, maneuvering battles where both sides aim for long-term positional advantages. White's main plan revolves around controlling the center, building up a kingside attack, or applying pressure on Black's queenside weaknesses. Meanwhile, Black often seeks counterplay through piece activity, pawn breaks (such as ...d5 or ...f5), and exploiting White's overextension.

This opening emphasizes deep positional understanding, as players must navigate complex pawn structures and subtle piece maneuvers. Its rich heritage includes famous games played by legends like Wilhelm Steinitz, Emanuel Lasker, and more recently, Magnus Carlsen. The Closed Ruy Lopez is an excellent choice for players who enjoy strategic battles and are willing to invest in learning its intricate plans and ideas.

The basic theme of the Closed Ruy Lopez is Whites's attempt to maintain his central pawn duo on d4 and e4. Black has various ways to combat the enemy pawn center, such as ...c5, as in the Chigorin line, or pressure against e4 (\variation{9...Bb7 10. d4 Re8}), as in the Zaitsev line. White shoud only resolve the tension in the centre by d5 or dxe5 under two circumstances: if he is forced to do so, or if he can gain a clear advantage by doing so \cite{book:nunn_grandmaster_move_by_move}.


