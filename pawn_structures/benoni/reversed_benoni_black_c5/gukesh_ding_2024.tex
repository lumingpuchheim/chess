\minisec{Gukesh--Ding,2024}
This was the 11th game of the World Championship Match. Both players have 5 points after 10 games and there
were four more to go. 

\newgame
\newchessgame[
id=A,
moveid=1w]
\mainline{
1. Nf3 d5 2. c4 d4 3. b4 c5 4. e3 Nf6 5. a3 Bg4 6. exd4 cxd4 7. h3 Bxf3 8. Qxf3
Qc7 9. d3 }

Strictly the only move to give White a comfortable edge is 9.c5!, but instead Gukesh played 9.d3?, having thought for just five minutes in a position where he had over an hour's lead on the clock. Why so fast? He revealed in the press conference that he'd thought he had the position on the board in the morning and that d3 was the move, but his guess was that he'd actually been looking at the position after 8...Nc6.

\mainline{9...a5 10. b5 Nbd7 11. g3}

\chessboard

With White's e-pawn exchanged with Black's c-pawn on d4, we have a (reversed) Benoni-like structure again. As in Benoni structure, White's d3 pawn is weak and 
Black's plan is often attacking the d3 pawn with his knight on c5. The next move is quite logical. 

\mainline{11...Nc5}! e5 is also good, a standard move to a standard problem (Benoni), what can go wrong? \mainline{12. Bg2 Nfd7} Move the knight to better e5 square. A typical Benoni play \variation{12... e5 13. O-O Bd6 } is also good since white needs to seek equality as in Benoni Defence. \mainline{ 13. O-O Ne5 14. Qf4 Rd8 15. Rd1}

\chessboard

Black has a much better position here: his pieces are better developed while White is still struggling to develop his queen side pieces because his d3 pawn is attacked.

With such a strategic advantage, it is useful to think in prophylaxis. Finding out what White tries to achieve in such desperate position helps to find the next move.

\begin{itemize}
    \item{White would love to get rid of the c5 knight with a4, \symbishop{} a3}
    \item{b6 would be a resource}
    \item{Black queen on c7 hangs}
\end{itemize}

Another method to find the next move is think schematically. Black wants to develop his bishop. The bishop would love to be on d6 to protect the Black queen and/or attack the White queen. e6 and then \symbishop d6 is logical.

\mainline{15...g6}?  

Alas, Ding lost the thread. \variation{15...e6} should have been played. For example \variation{16. b6 Qd6 17. Nd2 Nexd3 18. Qxd6 Rxd6} Black is up a pawn while White has little compensation.

Eventually Ding loses the important game. 
\mainline{16. a4 h5 17. b6  Qd6 18. Ba3 Bh6
19. Bxc5 Qxc5 20. Qe4 Nc6 21. Na3 Rd7 22. Nc2 Qxb6 23. Rab1 Qc7 24. Rb5 O-O 25.
Na1 Rb8 26. Nb3 e6 27. Nc5 Re7 28. Rdb1 Qc8 29. Qxc6 } 1-0
