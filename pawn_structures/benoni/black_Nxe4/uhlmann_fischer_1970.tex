\minisec{Uhlmann--Fischer,1970}
\newgame
\newchessgame[
id=A,
moveid=1w]
\mainline{1. d4 Nf6 2. c4 c5 3. d5 e6 4. Nc3 exd5 5. cxd5 d6 6. e4 g6
7. Bf4 a6 8. a4 Bg7 9. Nf3 O-O 10. Be2 Bg4 11. O-O Re8 12. h3
Nxe4 13. Nxe4 Rxe4 14. Bg5 Qe8 15. Bd3 Bxf3 16. Qxf3 Rb4
17. Rae1 Be5 18. Qd1 Qxa4 19. Qxa4 Rxa4 20. f4 Bd4+ 21. Kh1
Nd7 22. Re7 Nf6 23. Rxb7 Nh5 24. Kh2 Be3 25. Be2 Bxf4+
26. Bxf4 Rxf4 27. Rb6 Rxf1 28. Bxf1 Rd8 29. Bxa6 Kg7 30. Bb5
Kf6 31. Bc6 Ke5 32. Rb7 Rf8 33. Re7+ Kd4 34. Rd7 Nf6 } 

\chessboard

\mainline{15.Bh2}?!
We have a typical Benoni structure. e5 is out of the question because the Black has overprotected the square. 
White should have played a5 to dampen Black's queen side play with b5. 

\mainline{15...Rac8 16.Bc4 Ne5 17.f4 Nxc4
18.Qxc4 Nd7 19.Rfe1 Qb6 20.Rab1 Qb4}

White could not exchange the queen because
\variation{21. Qxb4 cxb4 22. Nb1 Nc5} White loses a pawn.

\mainline{21.Qf1 c4 22.Re2}?

This move is too passive. White should have played some king side assault, according to the Benoni spirit: \variation{
    22. f5 Ne5 23. f6 Bf8 24. Bxe5 Rxe5 25. Qf2 h5 
} the position is almost equal. After the text move, Black is simply better.

\mainline{22...b5
23.axb5 axb5 24.Kh1 Bxc3 25.bxc3 Qxc3 26.Rxb5 Qd3}

\chessboard

\mainline{27.Qe1?}

Last blunder. White should still have played according to Benoni spirit with \variation{27. e5 Nc5 28. Rb1 Qxd5 29. Rd1 Nd3 30. Bg3 dxe5 31. fxe5 h5 32. Re3}. The position is still holdable. 


\mainline{27...c3
28.Rb1 Nc5} 

0-1

\chessboard

White resigns because there is no way to stop the c-pawn without losing material.