The Open Sicilian is a dynamic and aggressive chess opening that arises after the:

\newgame
\newchessgame[
id=A,
moveid=1w,
]

\mainline{1. e4 c5 2. Nf3 d6 3. d4 cxd4 4. Nxd4}

\chessboard

This opening is one of the most popular and complex responses to 1. e4, leading to sharp, tactical battles where both sides have numerous opportunities for creative play. By choosing the Open Sicilian, White aims to seize central space and accelerate piece development, while Black counterattacks using active piece play and pawn breaks such as ...d5 or ...b5.

White has a lead in development and extra kingside space, which White can use to begin a kingside attack. This is counterbalanced by Black's central pawn majority, created by the trade of White's d-pawn for Black's c-pawn, and the open c-file, which Black uses to generate queenside counterplay and even a queenside attack if White decides to castle there.

\minisec{Black's Minority Attack}
The Sicilian Defence minority attack was known in the 1920s, but not so much so that books made mention of it. Later, with the Sicilian Defence explosion, that minority attack became one of the main reasons why White felt obliged to launch attacks at an early stage against Black's position \cite{book:secrets_of_modern_chess_strategy}.