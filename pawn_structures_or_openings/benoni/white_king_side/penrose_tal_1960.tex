A famous example of the e4-e5 break leading to a kingside attack occurred in the game. 
White installed a powerful knight on e4, while Black's pieces were hemmed in by the pawn on e5. Penrose soon crashed through on the f-file and scored a stunning upset over the reigning world champion.

\minisec{Penrose--Tal,1960}
\newgame
\newchessgame[
id=A,
moveid=1w]
\mainline{1.d4 Nf6 2.c4 e6 3.Nc3 c5 4.d5 exd5 5.cxd5 d6 6.e4 g6 7.Bd3
Bg7 8.Nge2 O-O 9.O-O a6 10.a4 Qc7 11.h3 Nbd7 12.f4 Re8 13.Ng3
c4 14.Bc2 Nc5 15.Qf3 Nfd7 16.Be3 b5 17.axb5 Rb8 18.Qf2 axb5
19.e5 dxe5 20.f5 Bb7 21.Rad1 Ba8 22.Nce4 Na4 23.Bxa4 bxa4
24.fxg6 fxg6 25.Qf7+ Kh8 26.Nc5 Qa7 27.Qxd7 Qxd7 28.Nxd7 Rxb2
29.Nb6 Rb3 30.Nxc4 Rd8 31.d6 Rc3 32.Rc1 Rxc1 33.Rxc1 Bd5
34.Nb6 Bb3 35.Ne4 h6 36.d7 Bf8 37.Rc8 Be7 38.Bc5 Bh4 39.g3} 

\chessboard