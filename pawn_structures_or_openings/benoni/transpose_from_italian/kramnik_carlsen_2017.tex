\minisec{Kramnik--Carlsen,2017}
\newgame
\newchessgame[
id=A,
moveid=1w]
\mainline{1. e4 e5 2. Nf3 Nc6 3. Bc4 Bc5 4. c3 Nf6 5. d3 d6 6. O-O a6 7. Re1 Ba7 8. a4 O-O 9. h3 Ne7 10. d4 Ng6 11. Nbd2 c6 12. Bd3 Re8 13. Bc2 h6 14. Nf1}

\storegame{game1}

\chessboard

Playing with Benoni is a huge commitment, because Benoni Defence is rarely played in high level games. \mainline{14...Bd7 15. a5 Bb8 16. Be3 Bc7 17. b4 d5 18. exd5 exd4 19. Bxd4 Nxd5 20. Qd2 Be6 21. Bxg6 fxg6 22. Re4 Nf6 23. Re2 Bd5 } gives Black a solid position. Rooks are likely to
be exchanged soon while Black's bishop pair can compensate the double g-pawn. 

\chessboard

\restoregame{game1}

Now we go back to the main game.

\chessboard

\mainline{14...exd4 15. cxd4 c5 16. d5 b5}

\chessboard

The position comes from Italian opening to Spanish then to Benoni. Some knowledge about Benoni structure may help to estimate this position:
\begin{itemize}
    \item{Black's dark square bishop moves from g7 to a more passive place}
    \item{For Black, c4 is premature because Black doesn't have enough compensation for the exchange sacrifice after \variation{18... c4 19. Be3 Qc7 20.Ra2 Qb7 21. Qa1 Bxe3 22. Rxa8} }
    \item{White has no f4, e5 play, because the Black's knights defend the king side well}
    \item{White cannot exploit the a-file because Black can simply exchange the heavy pieces}

\end{itemize}

From a human perspective, the position is still playable for both, because White has no easy targets/plans as in normal Benoni. 

\mainline{17. axb5 axb5 18. Ng3 Bd7 19. Be3 Bb6 20. Rxa8 Qxa8 21. b4 Qa7 22. Qa1 Qc7 23. Bxh6 cxb4 24. Bxg7 Qxc2 25. Qxf6 Qxf2+ 26. Kh2 Bd8 27. Qxd6 Nh4 28. Nxh4 Bxh4 29. Nh5 Bxh3 30. Rg1 Bg5 31. Bf6 Bg4 32. Bxg5 Bxh5 33. Qh6 Rxe4 34. Qxh5 Qf5 35. Qh6 b3 36. Bf6 Qf4+ 37. Qxf4 Rxf4 38. d6 Rxf6 39. Rd1 Rh6+ 40. Kg1 } 1-0

\chessboard